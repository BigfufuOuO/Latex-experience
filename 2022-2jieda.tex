\documentclass[11pt]{article}
\usepackage{ctex}
\usepackage{amsmath}
\usepackage{amsthm}
\usepackage{amssymb}
\usepackage{paralist}
\usepackage{enumerate}
\usepackage{tikz}
\usepackage{wrapfig}



\usepackage{geometry}
\geometry{left=2.0cm,right=2.0cm,top=2.5cm,bottom=2.5cm}




\title{\heiti 2022年新高考模拟2}
\author{\heiti 参考答案及参考评分标准}
\date{\today}
\begin{document}

	\maketitle

上一期中发现的错误:第2题的\heiti 解答 \songti,$\left | z-(a+b\mathrm{i} ) \right | $代表的是复数$z$到点$(a,b)$的距离. 第21题注记(2)中,切线的方程应该是$ \dfrac{1\cdot x}{4} +\dfrac{\dfrac{3}{2}\cdot x }{3} =1\Rightarrow x+2y=4 $.
\section{\heiti 单项选择题}
\begin{compactdesc}
	\item[解答1] B.
	\item[解答2] D.
	\item[解答3] B. 因为 $f(x)=a^{x}-2^{x}(a \neq 2)$ 为奇函数, 所以$f(x)+f(-x)=0, \quad a^{x}-2^{x}+a^{-x}-2^{-x}=0$,因而
	$\left(a^{x}-2^{x}\right)\left[1-\dfrac{1}{(2 a)^{x}}\right]=0$ 恒成立. 解得$(2 a)^{x}=1, a=\dfrac{1}{2}$.
	而$f(x)=2^{-x}-2^{x}$ 为 $\mathbf{R}$ 上的减函数, 且 $f(0)=0$,
	所以 $f(m)>0, m<0$, 则$m<-\dfrac{1}{2}\Rightarrow f(m)>0$,但反之不行.故 “ $m<-\dfrac{1}{2}$ ”是“ $f(m)>0$ ”的充分不必要条件.
	\item[解答4] C.
	\item[解答5] B.判断函数的奇偶性,再判断函数值的正负,从而排除错误选项,得正确选项.
	\item[解答6] C.先计算从夏至到冬至的晷长构成等差数列的公差和冬至到夏至的晷长构成等差数列的公差,再对选项各个节气对应的数列的项进行计算,判断说法的正误,即得结果.C错误,应为一丈一尺五寸.
	\item[解答7] C.由所给等式利用同角三角函数的关系可求得 $\cos \theta \cdot \sin \theta=\dfrac{1}{4}$, 再利用\heiti 降幂公式\footnote{其实就是二倍角公式的变形} \songti 及二倍角公式将 $\cos ^{2}\left(\theta+\dfrac{\pi}{4}\right)$ 整理为 $\dfrac{1-2 \sin \theta \cos \theta}{2}$, 或者利用$\cos ^2\left ( \theta +\dfrac{\pi }{4}  \right ) 
	=1-\sin ^2\left ( \theta +\dfrac{\pi }{4}  \right )
	=1-\left ( \dfrac{1}{\sqrt{2}}\left ( \sin x+\cos x \right )\right )^2 $化简也可.
		
		由 $\tan ^{2} \theta-4 \tan \theta+1=0$ 可得 $\tan \theta+\dfrac{1}{\tan \theta}=4$
		,所以 $\dfrac{\sin \theta}{\cos \theta}+\dfrac{\cos \theta}{\sin \theta}=4$, 即 $\dfrac{\sin ^{2} \theta+\cos ^{2} \theta}{\cos \theta \cdot \sin \theta}=4$, 即 $\cos \theta \cdot \sin \theta=\dfrac{1}{4}$ <
		$
		\cos ^{2}\left(\theta+\dfrac{\pi}{4}\right)=\dfrac{1+\cos \left(2 \theta+\dfrac{\pi}{2}\right)}{2}=\dfrac{1-\sin 2 \theta}{2}=\dfrac{1-2 \sin \theta \cos \theta}{2}=\dfrac{1-2 \times \dfrac{1}{4}}{2}=\dfrac{1}{4}
		$
		\begin{compactdesc}
			\item[\heiti 注意]千万不要尝试去解那个二次方程,它的根很奇怪.
		\end{compactdesc}
	\item[解答8]
		B. 从这十种乐器中挑八种全排列,有情况种数为$A_{10}^{8} $种.\\从除琵琶、二胡、编钟三种乐器外的七种乐器中挑五种全排列,有$A_{5}^{7}$种情况,再再从排好的五种乐器形成的6个空中挑3个插入琵琶、二胡、编钟三种乐器,有$A_{6}^{3}$种情况. 所以$P=\dfrac{A_{5}^{7}\cdot A_{6}^{3}}{A_{10}^{8}} =\dfrac{1}{6} $.
	\item[解答9]B. 第一次循环,$s=4,i=2$,第二次循环,$s=10,i=3$ ,第三次循环,$s=16,i=4$ ,结束循环,输出$s=16$.
		\begin{compactdesc}
			\item[\heiti 注意]写这种题,要像计算机一样思考。不然容易犯错,而且极其容易犯错,而且一旦犯错就发现自己真傻,拍大腿。因而要从初始条件一步一步来。如果步数太多,就先找规律快进到临界条件前,注意临界条件附近的变化,再从这里开始一步一步来直到跳出循环。不要想当然地跳出循环,否则你会感受无比后悔。
		\end{compactdesc}
	\item[解答10]
		 B.
		
		\begin{figure}[htbp]
			\centering
		\begin{minipage}{200pt}
			\centering
			\includegraphics[width=0.5\linewidth]{screenshot001}
			\caption{\heiti 第10题解1}
			\label{fig:t10}
		\end{minipage}
		\hspace{0pt}%
		\begin{minipage}{200pt}
			\centering
			\includegraphics[width=0.5\linewidth]{screenshot002}
			\caption{\heiti 第10题解2}
			\label{fig:t10_2}
		\end{minipage}
		\end{figure}	

	
		\heiti 标准解答:\songti 如图\ref{fig:t10},由题意 $\dfrac{\sqrt{3}}{4} A C^{2}=4 \sqrt{3}, A C=4$,取 $A B$ 中点 $O$ 为底面圆圆心, 连接 $O E, O D, D$ 为弧 $A B$ 中点, 则 $D O \perp A B$,	$\triangle A B C$ 是轴截面, 则平面 $A B C \perp$ 平面 $A B D$, 又 $D O \subset$ 平面 $A B D$, 平面 $A B C \cap$ 平面 $A B D$, 所以 $D O \perp$ 平面 $A B C$, 而 $O E \subset$ 平面 $A B C$, 所以 $O D \perp O E$,又 $E$ 是 $C B$ 中点, 则 $O E / / A C$,	所以 $\angle O E D$ 是异而直线 $A C, D E$ 所成角或其补角.且 $O E=\dfrac{1}{2} A C=2, O D=O A=2$, 所以 $\angle O E B=\dfrac{\pi}{4}, \cos \angle OED=\dfrac{\sqrt{2}}{2}$,	异而直线 $A C, D E$ 所成角的余弦值为 $\dfrac{\sqrt{2}}{2}$.
		
		\heiti 另一种解法:\songti 如图\ref{fig:t10_2},延长$BD$使得$ FA\perp AB $,此时$ AF=AC $,因而直角$\triangle CAF$是等腰直角三角形,故可以秒选B.
	
	\item[解答11]	
		A.由题意, $2 p=2$, 则 $\dfrac{p}{2}=\dfrac{1}{2}$, 故抛物线 $x^{2}=2 y$ 的焦点坐标是 $\left(0, \dfrac{1}{2}\right)$, 由抛 物线的定义得, 点 $P$ 到准线 $y=-\dfrac{1}{2}$ 的距离等于 $|P F|$, 即为 1 , 故点 $P$ 到直 线 $y=2$ 的距离为 $d=2-\left(-\dfrac{1}{2}\right)-1=\dfrac{3}{2}$. 设 点 $P$ 在直线 $y=2$ 上的射影为 $P^{\prime}$, 则 $\left|P P^{\prime}\right|=\dfrac{3}{2}$. 当点 $M, N$ 在 $P^{\prime}$ 的同一侧 (不与点 $P^{\prime}$ 重合) 时, $|P M|+|P N|+|M N|>\dfrac{3}{2}+2+\dfrac{5}{2}=6$, 不符合题意; 当点 $M, N$ 在 $P^{\prime}$ 的异侧(不与 点 $P^{\prime}$ 重合) 时, 不妨设 $\left|P^{\prime} M\right|=x(0<x<2)$, 则 $\left|P^{\prime} N\right|=2-x$, 故由 $|P M|+|P N|+|M N|=\sqrt{x^{2}+\left(\dfrac{3}{2}\right)^{2}}+\sqrt{(2-x)^{2}+\left(\dfrac{3}{2}\right)^{2}}+2=6$, 解得 $x=0$ 或 2 , 不 符合题意, 舍去, 综上 $M, N$ 在两点中一定有一点与点 $P^{\prime}$ 重合, 所以 $\sin \angle M P N=\dfrac{2}{\dfrac{5}{2}}=\dfrac{4}{5}$, 故选 A.
		\\ \heiti 评述:\songti 此题出的不好,需要考虑$M,N$是否在同一侧的问题,而在同一侧时莫名其妙地用了一个不等式(这里等号成立是$M$与$P^{\prime}$重合时得到),着实让人摸不着头脑。而且也没有考察很多的圆锥曲线知识,后半程都在分析奇怪的几何上去了,因而此题出的不好,当做乐呵乐呵得了.
		
	\item [解答12]
		C.由已知条件可推得$ f(x_1)=x_1+\ln (x_1-1)=1+2\ln t $,即$ x_1-1+\ln (x_1-1)=\ln t^2 $ ,∴$t^{2}=\left(x_{1}-1\right) e^{x_{1}-1}$,而$g(x_2)=x_2\ln x_2=e^{\ln x_{2}} \cdot \ln x_{2}=t^2$, 即有 $\ln x_{2}=x_{1}-1$, 结合目标式化简 可得 $\left(x_{1} x_{2}-x_{2}\right) \ln t=(x_1-1)x_2\ln t=t^{2} \cdot \ln t$, 令 $h(t)=t^{2} \cdot \ln t$, 利用导函数研究其单调性并确定区间 最小值, 即为 $\left(x_{1} x_{2}-x_{2}\right) \ln t$ 的最小值.
		\\不难计算得$h(t)_{\min}= h(e^{-\frac{1}{2} })=-\dfrac{1}{2e} $.		
\end{compactdesc}

\section{\heiti 填空题}
错填不得分. 
\begin{compactdesc}
	\item[解答13] \underline{ $\dfrac{\pi}{2}$ }. 
	\item[解答14] \underline{ $ \left ( \dfrac{1}{32},\dfrac{1}{16}   \right )  $ }. 根据函数解析式,作出函数图象,将方程有4个不等实数根,转化为函数$y=f(x)$的图象与直线$ y=a(x-1) $有四个不同的交点,利用数形结合的方法,即可求出结果.注意端点处是“空心”的还是“实心”的,即取不取得到的问题.
	\item[解答15]  \underline{ $y=3$ 或 $y=\dfrac{4}{3}x+3$ }. 若少情况则不得分. 解之:根据题意, 圆 $C$ 即 $(x-1)^{2}+(y-1)^{2}=8$, 圆心 $C(1,1)$, 半径 $r=2 \sqrt{2}$,
	又由直线 $l$ 与圆 $C$ 交于 $A 、 B$ 两点, $|A B|=4$, 则点 $C$ 到直线 $l$ 的距离
	$d=\sqrt{r^{2}-\left(\dfrac{|A B|}{2}\right)^{2}}=2	$. 
	\heiti 若直线 $l$ 的斜率不存在,\songti 直线 $l$ 的方程为 $x=0$, 点 $C$ 到直线 $l$ 的距离 $d=1$, 不符合题意; 若直线 $l$ 的斜率存在, 设直线 $l$ 的方程为 $y=k x+3$, 即 $k x-y+3=0$,则有 $d=\dfrac{|k+2|}{\sqrt{1+k^{2}}}=2$, 解可得 $k=0$ 或 $\dfrac{4}{3}$; 故直线 $l$ 的方程为 $y=3$ 或 $y=\dfrac{4}{3} x+3$;
	\\ \heiti 评注:\songti 一定要养成考虑斜率不存在的习惯,无论是小题还是大题,否则很容易漏答案.
	
	\item [解答16] \underline{ $ \dfrac{211}{256} $ }. 
	解: 记 $M$ 为 $A C$ 的中点, 由中线长公式得$
	4 B M^{2}+A C^{2}=2\left(A B^{2}+B C^{2}\right)$,
	可得 $A C=\sqrt{2\left(6^{2}+4^{2}\right)-4 \cdot 10}=8$.
	$$
	\begin{aligned}
		\text { 由余弦定理得 } \cos A &=\frac{C A^{2}+A B^{2}-B C^{2}}{2 C A \cdot A B}=\frac{8^{2}+6^{2}-4^{2}}{2 \cdot 8 \cdot 6}=\frac{7}{8}, \text { 所以 } \\
		\sin ^{6} \frac{A}{2}+\cos ^{6} \frac{A}{2} &=\left(\sin ^{2} \frac{A}{2}+\cos ^{2} \frac{A}{2}\right)\left(\sin ^{4} \frac{A}{2}-\sin ^{2} \frac{A}{2} \cos ^{2} \frac{A}{2}+\cos ^{4} \frac{A}{2}\right) \\
		&=\left(\sin ^{2} \frac{A}{2}+\cos ^{2} \frac{A}{2}\right)^{2}-3 \sin ^{2} \frac{A}{2} \cos ^{2} \frac{A}{2} \\
		&=1-3\cdot \left ( \frac{1}{2} \sin A \right ) ^2=1- \frac{3}{4} \sin ^{2} A \\
		&=\frac{1}{4}+\frac{3}{4} \cos ^{2} A=\frac{211}{256} .
	\end{aligned}
	$$
	\\ \heiti 评注:\songti 实际上,在解此题时,应该先把$ \sin ^{6} \dfrac{A}{2}+\cos ^{6} \dfrac{A}{2} $化为$ \dfrac{1}{4}+\dfrac{3}{4} \cos ^{2} A $,进而去寻找$ \cos A$. 而要计算之,由余弦公式则需要先计算$AC$的值,再回推去找$AC$的值,蕴含在中线长公式里,故一步一步可推出答案.此题来自于2020年全国高中数学联赛·决赛填空题第5题. 
	\begin{proof}[\heiti 中线长公式定理的证明]
		设$ \triangle ABC $中,$BC$边的中点为$D$,则中线为$AD$.
		\\而由向量知识可知,$ \overrightarrow{AD}=\dfrac{1}{2}\left ( \overrightarrow{AB}+\overrightarrow{AC}   \right ) $. 两边平方整理得
		$$ 4\left | \overrightarrow{AD} \right |^2 =
		\left | \overrightarrow{AB} \right |^2 +\left | \overrightarrow{AC}  \right |^2
		+2 \overrightarrow{AB}\cdot \overrightarrow{AC} $$
		而$ \overrightarrow{AB}\cdot \overrightarrow{AC}=
		\dfrac{\left ( \overrightarrow{AB}+\overrightarrow{AC} \right )^2-\left ( \overrightarrow{AB}-\overrightarrow{AC} \right )^2}{4}
		=\dfrac{\left |2 \overrightarrow{AD}\right |^2-\left | \overrightarrow{BC} \right |^2}{4}   $,代入整理即得
		$$
		\left | \overrightarrow{AD}  \right |^2
		=\dfrac{1}{2}  \left | \overrightarrow{AB}  \right |^2
		+\dfrac{1}{2}  \left | \overrightarrow{AC}  \right |^2
		-\dfrac{1}{4}  \left | \overrightarrow{BC}  \right |^2
		$$
		即$ 4AD^2=2(AB^2+AC^2)-BC^2 $.
	\end{proof}	
	要熟记这个公式,它给出了\heiti 一条边对应的中线\footnote{中线是对应一条边而言的. 如这里$BC$边对应的中线就是$AD$.}\songti 与三角形三边之间的关系.
\end{compactdesc}

\section{\heiti 解答题}
\subsection{\heiti 必做题}
\heiti 解答17    \songti 【答案】见下.
\\(1) \fangsong 利用等差数列的性质 $a_{n-k}+a_{n+k}=2 a_{n}$, 配出 “$P (3) $数列” 所满足的条件即可. \songti \\
因为 $\left\{a_{n}\right\}$ 是等差数列, 所以, 当 $n \geqslant 4$ 时, $a_{n-3}+$ $a_{n+3}=2 a_{n}, a_{n-2}+a_{n+2}=2 a_{n}, a_{n-1}+a_{n+1}=2 a_{n}$,\dotfill{2分}\\各式 相加可得 $a_{n-3}+a_{n-2}+a_{n-1}+a_{n+1}+a_{n+2}+a_{n+3}=$ $6 a_{n}$, 因此, 等差数列 $\left\{a_{n}\right\}$ 是 $P(3)$ 数列.\dotfill{4分}
\\(2) \fangsong 利用 “ $P(k)$ 数列” 的定义、等差数列的定义以 及通项公式、数列的递推关系即可解决问题.
\\ \songti 因为数列 $\left\{a_{n}\right\}$ 既是 “ $P$ (2) 数列”, 又是 “ $P$ (3) 数 列”, 因此, 当 $n \geqslant 3$ 时,
\begin{equation}
	a_{n-2}+a_{n-1}+a_{n+1}+a_{n+2}=4 a_{n} \label{1}
\end{equation}
当$n \geqslant 4$时, 
\begin{equation}	
	a_{n-3}+a_{n-2}+a_{n-1}+a_{n+1}+a_{n+2}+a_{n+3}=6 a_{n} \label{2}
\end{equation}
由式(1)递推, 分别令$n=n-1$和$n=n+1$替换$n$,得
\begin{align}
	&a_{n-3}+a_{n-2}=4 a_{n-1}-\left(a_{n}+a_{n+1}\right), \label{3}\\
	&a_{n+2}+a_{n+3}=4 a_{n+1}-\left(a_{n-1}+a_{n}\right) \label{4}
\end{align}
\dotfill{8分}
\\
\kaishu 其中每个式子1分. 若有其他方法,正确地得出下面一步的结论,也能获得此步骤分. 但是列出了式子(1)、(2),直接相减得$a_{n-3}+a_{n+3}=2 a_{n}$并直接下结论的,只能获得6分. \songti
\\
把式(3)、(4) 代入 (2) 得 $a_{n-1}+a_{n+1}=2 a_{n}$, \dotfill{9分}
\\其中 $n \geqslant$ 4. 所以 $a_{3}, a_{4}, a_{5}, \cdots$ 是等差数列,设其公差为 $d$.  \dotfill{10分}
\\ \kaishu 若直接由$a_{n-1}+a_{n+1}=2 a_{n}$,不说明$n \geqslant 4$ ,就直接下结论$\left\{a_{n}\right\}$ 是等差数列的,到此结束,只能获得9分. 只要由$a_{n-1}+a_{n+1}=2 a_{n}$能够判断出 $a_{3}, a_{4}, a_{5}, \cdots$ 是等差数列,即得10分.\songti
\\
在式(1)中取 $n=4$, 则 $a_{2}+a_{3}+a_{5}+a_{6}=4 a_{4}$, 故 $a_{2}+a_{3}+\left(a_{3}+2 d\right)+\left(a_{3}+3 d\right)=4\left(a_{3}+d\right)$, 即 $a_{3}=$ $a_{2}+d$. 从而 $a_{2}, a_{3}, a_{4}, a_{5}, \cdots$ 是公差为 $d$ 的等差数列, \dotfill{11分}
\\在式(1)中取 $n=3$, 则 $a_{1}+a_{2}+a_{4}+a_{5}=4 a_{3}$, 故 $a_{1}+a_{2}+\left(a_{2}+2 d\right)+\left(a_{2}+3 d\right)=4\left(a_{2}+d\right)$,
即 $a_{2}=a_{1}+d$.
从而, $a_{1}, a_{2}, a_{3}, a_{4}, a_{5}, \cdots$ 是公差为 $d$ 的等差数 列, 即 $\left\{a_{n}\right\}$ 是等差数列. \dotfill{12分}

\heiti 评注:\songti 此题主要利用新定义考察旧知识,而其中包含的一些小坑,如$n$的范围需要特别注意. 

\heiti 解答18 \songti 【答案】(1)见下;(2)$ \dfrac{\sqrt{3}}{2} $.
\\(1)\fangsong 根据 $B C / /$ 平面 $\alpha$, 易知 $F$ 为棱 $P B$ 的中点, 再根据 $B C \perp A B$, 得到 $E F \perp A B$,
再由 $P A \perp$ 平面 $A B C$, 得 $P A \perp A B$, 然后由线面垂直的判定定理证明.\\
\songti 因为 $B C / /$ 平面 $\alpha, B C \subset$ 平面 $P B C$, 平面 $\alpha \cap$ 平面 $P B C=E F$,
\\所以 $B C / / E F$, 且 $F$ 为棱 $P B$ 的中点.\dotfill{1分}\\
因为 $B C \perp A B$, 所以 $E F \perp A B$.\dotfill{2分}\\
因为 $P A / /$ 平面 $\alpha, P A \subset$ 平面 $P A C$, 平面 $\alpha \cap$ 平面 $P A C=D E$,
所以 $P A / / D E$.\dotfill{3分}\\
因为 $P A \perp$ 平面 $A B C$,
所以 $P A \perp A B$,故 $D E \perp A B$, \dotfill{4分}
\\又 $D E \cap E F=E$, $DE \subset$ 平面$\alpha$,$EF \subset$平面$\alpha$
所以 $A B \perp$ 平面 $D E F$,
即 $A B \perp$ 平面 $\alpha .$.\dotfill{6分}
\\ \kaishu 条件缺一不可,否则不得对应步骤分.
\\ \songti (2)如图所示, 以点 $B$ 为坐标原点, 分别以 $B A, B C$ 所在直线为 $x, y$ 轴, 过点 $B$ 且 与 $A P$ 平行的直线为 $z$ 轴建立空间直角坐标系,
% TODO: \usepackage{graphicx} required
\begin{figure}[htbp]
	\centering
	\includegraphics[width=0.3\linewidth]{screenshot004}
	\caption{\heiti 第18题解}
	\label{fig:screenshot004}
\end{figure}
\\则 $B(0,0,0), A(2,0,0), C(0,2,0), P(2,0,2), E(1,1,1), F(1,0,1)$,
$\overrightarrow{A C}=(-2,2,0), \overrightarrow{B C}=(0,2,0), \overrightarrow{B P}=(2,0,2)$, \dotfill{7分}
\\设 $M(1, t, 1)$,
则 $\overrightarrow{A M}=(-1, t, 1)$, 
\\设平面 $M A C$ 的一个法向量为 $\boldsymbol{m}=\left(x_{1}, y_{1}, z_{1}\right)$,
则 $\left\{\begin{array}{l}\boldsymbol{m} \cdot \overrightarrow{A C}=-2 x_{1}+2 y_{1}=0, \\ \boldsymbol{m} \cdot \overrightarrow{AM}=-x_{1}+t y_{1}+z_{1}=0,\end{array}\right.$ 令 $x_{1}=1$, 则 $y_{1}=1, z_{1}=1-t$,
取 $\boldsymbol{m}=\left(1, 1, 1-t\right)$ \dotfill{8分}
\\
设平面 $P B C$ 的一个法向量为 $\boldsymbol{n}=\left(x_{2}, y_{2}, z_{2}\right)$,
则 $\left\{\begin{array}{l}\boldsymbol{n} \cdot \overline{B C}=2 y_{2}=0, \\ \boldsymbol{n} \cdot \overline{B P}=2 x_{2}+2 z_{2}=0,\end{array}\right.$
则 $y_{2}=0$, 令 $x_{2}=1$, 则 $z_{2}=-1$,
取$\boldsymbol{n}=(1,0,-1)$ .\dotfill{9分}
\\设平面 $M A C$ 与平面 $P B C$ 所成的锐二面角为 $\theta$,
则 $$\cos \theta=|\cos \langle\boldsymbol{m}, \boldsymbol{n}\rangle|=
\dfrac{|\boldsymbol{m} \cdot \boldsymbol{n}|}{|\boldsymbol{m}||\boldsymbol{n}|}=\dfrac{|1-(1-t)|}{\sqrt{1^{2}+1^{2}+(1-t)^{2}} \times \sqrt{2}}=\dfrac{|t|}{\sqrt{t^{2}-2 t+3} \times \sqrt{2}}$$ 
\\即$\cos \theta=\dfrac{1}{\sqrt{\dfrac{3}{t^{2}}-\dfrac{2}{t}+1} \times \sqrt{2}}=\dfrac{1}{\sqrt{3\left(\dfrac{1}{t}-\dfrac{1}{3}\right)^{2}+\dfrac{2}{3} \times \sqrt{2}}}$,\dotfill{10分}
\\当且仅当 $\dfrac{1}{t}=\dfrac{1}{3}$, 即 $t=3$ 时, $3\left(\dfrac{1}{t}-\dfrac{1}{3}\right)^{2}+\dfrac{2}{3}$ 取得最小值 $\dfrac{2}{3}$,
$\cos \theta$ 取得最大值为 $\dfrac{1}{\sqrt{\dfrac{2}{3}} \times \sqrt{2}}=\dfrac{\sqrt{3}}{2}$.\dotfill{11分}
\\所以平面 $M A C$ 与平面 $P B C$ 所成锐二面角的余弦值的最大值为 $\dfrac{\sqrt{3}}{2}$.\dotfill{12分}

\heiti 解答19 \songti 【答案】$\dfrac{3}{2}$.
\\ \heiti 方法1: 
\\ \songti 设$ P(x_0,y_0)(y_0 \neq 0) $,则$ y_0^2=2px_0 $.\dotfill{1分}
\\ 对$y^2=2px$,两边对$x$求导得$2yy'=2p$,所以$y'=\dfrac{p}{y}$.
故过$P$点的切线方程为$y-y_0=\dfrac{p}{y}(x-x_0)$,整理得$ y_0y-y_{0}^{2}=p(x-x_0) $,代入$ y_0^2=2px_0$得切线方程$yy_0=p(x+x_0)$.\dotfill{4分}
\\ 令$x=0$得$Q\left ( 0,\dfrac{px_0}{y_0}  \right ) $.
\\又$ F\left ( \dfrac{p}{2},0  \right ) $,由$ \left | FP \right |=2  $得$ \sqrt{\left ( x_0-\dfrac{p}{2}  \right )^2+y_0^2 } =2 $.
\\由$ \left | FQ \right |=1  $得$ \sqrt{\left ( \dfrac{p}{2}  \right )^2 +\left ( \dfrac{px_0}{y_0}  \right ) ^2 } =1 $.\dotfill{7分} 
\\整理两式分别得到
\begin{align}
	\left ( x_0-\dfrac{p}{2}  \right )^2+2px_0&=4  \\
	 \dfrac{1}{4}p^2+\dfrac{x_0^2}{y_0^2}&=1  
\end{align}
\\将$y_0^2=2px_0$代入(6),并将$(5)-(6)$得$x_0^2+\dfrac{p}{2}x_0=3$.
由(6)式又得$x_0=\dfrac{4-p^2}{2p}$,代入前式,得
\begin{equation}
	\left ( \dfrac{4-p^2}{2p}  \right ) ^2+\dfrac{p}{2}\cdot \dfrac{4-p^2}{2p}=3  
\end{equation}
令$m=p^2 $化简得$4m=4$,所以$m=1$,故$p=1$.\dotfill{10分}
\\ 所以解得$x_0=\dfrac{3}{2}$.
故$\overrightarrow{OP}\cdot \overrightarrow{OQ}=0+\dfrac{x_0}{y_0} \cdot y_0=x_0=\dfrac{3}{2} $. \dotfill{12分}
\\ \kaishu 若这里求出了$y_0=\pm \sqrt{3}$也可得. 然而,若只求出$ y_0=\sqrt{3} $或$-\sqrt{3}$,存在漏解,扣1分.
\\ \heiti 方法2:
\\ \songti 设 $P\left(\dfrac{t^{2}}{2 p}, t\right)(t \neq 0)$, 则 $\Gamma$ 的切线 $P Q$ 的方程为 $y t=p\left(x+\dfrac{t^{2}}{2 p}\right)$. \dotfill{4分} \\ \kaishu 同方法1,要写出求切线的步骤,否则该4分只能得1分.
\\ \songti 令 $x=0$,得 $y=\dfrac{t}{2}$, 故 $Q\left(0, \dfrac{t}{2}\right)$. 
\\又 $F$ 坐标为 $\left(\dfrac{p}{2}, 0\right)$, 进而
$$
|F P|=\sqrt{\left(\dfrac{p}{2}-\dfrac{t^{2}}{2 p}\right)^{2}+t^{2}}=\dfrac{p}{2}+\dfrac{t^{2}}{2 p},|F Q|=\dfrac{\sqrt{p^{2}+t^{2}}}{2},
$$
\dotfill{7分} \\
结合 $|F P|=2,|F Q|=1$ 可分别得 $p^{2}+t^{2}=4 p, p^{2}+t^{2}=4$. 所以 $p=1, t^{2}=3$. \dotfill{10分}
\\于是 $\overrightarrow{O P} \cdot \overrightarrow{O Q}=\dfrac{t^{2}}{2}=\dfrac{3}{2}$.\dotfill{12分}
\\ \heiti 评注:\songti 本题提醒我们,参数方程能够省略许多计算步骤,而且是被允许的,因而有时候学会使用参数方程解圆锥曲线大题十分有必要.

\heiti 解答20 \songti 【答案】(1)140;(2)i. $ p_0=\dfrac{3}{4} $;ii.见下.
\\ (1)\fangsong 由正态分布$ 3\sigma   $原则即可求出排球个数;
\\ \songti (1) 因为 $\xi$ 服从正态分布 $N\left(270,5^{2}\right)$, 所以
$P(260<\xi \leqslant 265)=P(\mu -2\sigma <\xi<\mu+2\sigma )-P(\mu-\sigma <\xi<\mu-\sigma )=\dfrac{0.9644-0.6826}{2}=0.1409,$ \dotfill{2分}
\\所以质量指标在 $(260,265]$ 内的排球个数为 $1000 \times 0.1409=140.9 \approx 140$ 个;\dotfill{3分}
\\ \kaishu 若这里算得141个,只能得2分. 因为141个是取不到的,根据正态分布图可知,若要取到141个,则已经超出了$260<\xi\leqslant265$所限定的范围.
\\
(2)\songti 
\begin{enumerate}
	\item[i.] $f(p)=C_{3}^{1} p^{3}(1-p)=3 p^{3}(1-p)$\dotfill{5分}
	\\ \heiti 解法1:\songti \\$f^{\prime}(p)=3\left[3 p^{2}(1-p)+p^{3} \times(-1)\right]=3 p^{2}(3-4 p)$.
	\\令 $f^{\prime}(p)=0$, 得 $p=\dfrac{3}{4}$,
	\\当 $p \in\left(0, \dfrac{3}{4}\right)$ 时, $f^{\prime}(p)>0, f(p)$ 在 $\left(0, \dfrac{3}{4}\right)$ 上单调递增;
	当 $p \in\left(\dfrac{3}{4}, 1\right)$ 时, $f^{\prime}(p)<0, f(p)$ 在 $\left(\dfrac{3}{4}, 1\right)$ 上单调递减;
	\\所以 $f(p)$ 的最大值点 $p_{0}=\dfrac{3}{4}$;\dotfill{7分}
	\\ \heiti 解法2:\songti
	\\ 由均值不等式,$3p^3(1-p)=p^3(3-3p)\leqslant \left ( \dfrac{p+p+p+3-3p}{4}  \right )^4=\left ( \dfrac{3}{4}  \right )^4  $,当且仅当$3-3p=p$,即$p=\dfrac{3}{4}$取到. 
	\\故$p_0=\dfrac{3}{4}$. \dotfill{7分}
	\\ \kaishu 不写取等条件不扣分,因为答案即为取等. 当然,最好还是写上以养成习惯.
	\\ \fangsong 由于题目所说是“最大值点”,故可用不等式,若说“极大值点”,最好还是用导数.
	\item[ii.] \fangsong 先列出所有可能的情况:0:3、1:3、2:3、3:2、3:1、3:0,依次所得的分数是0、0、1、2、3、3.
	\\ \songti $X$的可能取值为0,1,2,3.
	\\ $ P(X=0)=(1-p_0)^3+C_3^1p_0(1-p_0)^3=\dfrac{13}{256}  $,
	\\ $ P(X=1)=C_4^2p_0^2(1-p_0)^3=\dfrac{27}{512}  $,
	\\ $ P(X=2)=C_4^2p_0^3(1-p_0)^2=\dfrac{81}{512}  $,
	\\ $ P(X=3)=p^3+C_3^2p_0^3(1-p_0)=\dfrac{189}{256}  $.\dotfill{11分}
	\\ \kaishu 每一种情况1分. \songti
	\\ 故X的分布列为:
	\begin{table}[htbp]
		\centering
		\begin{tabular}{c|c|c|c|c}
			\hline
			$X$ & 0 & 1 & 2 & 3 \\
			\hline \rule{0pt}{20pt}
			$p$ & $ \dfrac{13}{256}  $ & $ \dfrac{27}{512} $ & $ \dfrac{81}{512} $ & $ \dfrac{189}{256}  $ \\
			\hline
		\end{tabular}
	\end{table}
	\dotfill{12分}
\end{enumerate}
\heiti 评注:\songti 这种五局三胜制的比赛,需要注意如果任意一方只要到达3局比赛就结束了,因而3:1不可能出现胜胜胜负的情况,3:2不可能出现胜负胜胜负的情况. 这时候如果用插入法,以下黑圈代表胜利,白圈代表失败。先把胜利摆成一排:
	\begin{center}
		\tikz \fill(0,0) circle(1ex);  \tikz \fill(0,0) circle(1ex);   \tikz \fill(0,0) circle(1ex); 
	\end{center}
	然后让白球插入其中的挡板,可以插在黑球前面,但是注意不能让三个黑球一起出现在最前面,这样直接变成三胜. 所以直插入一个白球时,只能插在如图所示的位置:
	\begin{center}
		$^\vee $ \tikz \fill(0,0) circle(1ex); $^\vee $ \tikz \fill(0,0) circle(1ex);  $^\vee $ \tikz \fill(0,0) circle(1ex); 
	\end{center}
故3:1时为$C_3^1$. 而3:2的情况则有些不同,插入一个新的白球后,会多出一个空,此时再插入一个白球,这里出现了分步,人为地排列了白球,而两个白球是等价的,只需组合,因而要消去这里的分步,故所得种数为$ \dfrac{C_3^1\cdot C_4^1}{A_2^2}=6 $.
\\换另一个角度思考,既然五局三胜的胜方最后一局必定是胜利的,我们可以忽略最后一局,所以对于3:1,考虑前三把,胜方只输了一把,任意一把均可,故种数为$ C_3^1 $;对于3:2,考虑前四把,胜方只输了两把,任意两把均可,故种数为$C_4^2$.


\heiti 解答21 \songti 【答案】(1)2;(2)见下.
\\(1)解因为 $f(x)=\ln (x+3)-x$, 所以 $f^{\prime}(x)=\dfrac{-x-2}{x+3}(x>-3)$.\dotfill{1分}
\\令 $f^{\prime}(x)>0$ , 得 $-3<x<-2$ ; 令 $f^{\prime}(x)<0$ ,得 $x>-2$ ,
\\所以 $f(x)$ 在 $(-3,-2)$ 上单调递增, 在 $(-2,+\infty)$ 上单调递减,\dotfill{3分}
\\所以 $f(x)_{\max }=f(-2)=2$.\dotfill{4分}
\\(2)方程可化为 $e^{x+\ln a}+x+\ln a=x+3+$ $\ln (x+3)=e^{\ln (x+3)}+\ln (x+3)$
\\设 $g(x)=e^{x}+x$,  $g(x)$ 在 $(-\infty,+\infty)$ 上是增函数, 又 $g(x+\ln a)=g(\ln (x+3))$,
\\所以有 $x+\ln a=\ln (x+3)$ , 即方程 $\ln (x+3)-x=\ln a$ 有两个实数根 $x_{1},x_{2}$.\dotfill{7分}
\\因为方程 $f(x)=\ln a$ 有两个实数根 $x_{1},x_{2}$ ,所以 $\left\{\begin{array}{l}\ln \left(x_{1}+3\right)=x_{1}+\ln a \\ \ln \left(x_{2}+3\right)=x_{2}+\ln a\end{array}\right.$, \dotfill{8分}
\\则
$\dfrac{\left(x_{1}+3\right)-\left(x_{2}+3\right)}{\ln \left(x_{1}+3\right)-\ln \left(x_{2}+3\right)}=1$, \dotfill{9分}
\\而
$a e^{x_{1}}+a e^{x_{2}}=e^{x_{1}+\ln a}+e^{x_{2}+\ln a}=e^{\ln \left(x_{1}+3\right)}+e^{\ln \left(x_{2}+3\right)}=x_{1}+x_{2}+6$, 
\\故需证 $x_{1}+3+x_{2}+3>2$,即证 $x_{1}+3+x_{2}+3>\dfrac{2\left(x_{1}+3\right)-2\left(x_{2}+3\right)}{\ln \left(x_{1}+3\right)-\ln \left(x_{2}+3\right)}$. 
\\不妨设 $-3<x_{1}<x_{2}$, 令 $t=\dfrac{x_{1}+3}{x_{2}+3}$, 则
$0<t<1$, 即要证 $\ln t<\dfrac{2(t-1)}{t+1}(0<t<1)$.
\\设 $h(t)=\ln t-\dfrac{2(t-1)}{t+1}(0<t<1)$, 则 $h^{\prime}(t)=\dfrac{(t-1)^{2}}{t(t+1)^{2}}>0$, 所以 $h(t)$ 在 $(0,1)$ 上是增函
数, $h(t)<h(1)=0$, 即 $\ln t<\dfrac{2(t-1)}{t+1}$ 成立, 故原式成立.\dotfill{12分}

\subsection{选做题}
\heiti 解答22 \songti 【答案】见下.
\\(1)因为 $C_{2}:\left\{\begin{array}{l}x=\dfrac{\sqrt{3}}{2} \cos \varphi, \\ y=\dfrac{\sqrt{3}}{2}+\dfrac{\sqrt{3}}{2} \sin \varphi,\end{array}\right.$ 
所以曲线 $C_{2}$ 的普通方程为: $x^{2}+\left(y-\dfrac{\sqrt{3}}{2}\right)^{2}=\dfrac{3}{4}$, 由 $\left\{\begin{array}{l}x=\rho \cos \theta, \\ y=\rho \sin \theta,\end{array}\right.$ 得曲线 $C_{2}$ 的极坐标方程 $\rho=\sqrt{3} \sin \theta$.\dotfill{3分}
\\对于曲线 $C_{1}: \dfrac{x^{2}}{4}+y^{2}=1$,令$\left\{\begin{array}{l}x=\rho \cos \theta, \\ y=\rho \sin \theta,\end{array}\right.$ ,则曲线 $C_{1}$ 的极坐标方程为 $\rho^{2}=\dfrac{4}{1+3 \sin ^{2} \theta}$.\dotfill{5分}
\\(2) 由(1)得 $|O A|^{2}=\rho^{2}=\dfrac{4}{1+3 \sin ^{2} \alpha},|O B|^{2}=\rho^{2}=3 \sin ^{2} \alpha$, 
\\$|O A|^{2}+|O B|^{2}=\dfrac{4}{1+3 \sin ^{2} \alpha}+3 \sin ^{2} \alpha=\dfrac{4}{1+3 \sin ^{2} \alpha}+\left(3 \sin ^{2} \alpha+1\right)-1 .$ \dotfill{7分}
\\令 $t=1+3 \sin ^{2} \alpha$, 则 $t \in\left[ \dfrac{13}{4},4\right ),$ $f(t)=|O A|^{2}+|O B|^{2}=\dfrac{4}{t}+t-1$, \dotfill{8分}
\\又 $f(t)$ 在 $\left[\dfrac{13}{4}, 4\right)$ 单调递增,
而 $f\left(\dfrac{13}{4}\right)=\dfrac{181}{52}, f(4)=4$, 所以 $|O A|^{2}+|O B|^{2} \in\left[\dfrac{181}{52}, 4\right)$.\dotfill{10分}
\\ \heiti  官方评注:\songti 求曲线 $C_{2}$ 的极坐标方程要经过两次转化, 问题 (2) 转化为三角函数的值域, 再转化为 $t \in\left[\dfrac{13}{4}, 4\right)$, 时, 求 $y=\dfrac{4}{t}+t-1$ 的值域. 而该函数在 $\left[\dfrac{13}{4}, 4\right)$ 单调递增, 值域便可求出.

\heiti 解答23 \songti 【答案】见下
\\(1)由题,两边平方得$ a+b+2\sqrt{ab}>c+d+2\sqrt{cd}   $,因为$ a+b=c+d $,而由$ ab>cd $得$ \sqrt{ab}>\sqrt{cd} $,故原不等式成立.\dotfill{2分}
\\(2)证明必要性:
\\ 若$ \left | a-b \right |>\left | c-d \right |   $,则$ (a-b)^2<(c-d)^2 $,即$ (a+b)^2-4ab<(c+d)^2-4cd $\dotfill{4分}
\\因为$ a+b=c+d $,所以$ ab>cd$.
\\由(1)知$ \sqrt{a}+\sqrt{b}>\sqrt{c}+\sqrt{d} $.\dotfill{6分}
\\证明充分性:
\\若 $\sqrt{a}+\sqrt{b}>\sqrt{c}+\sqrt{d}$, 则 $(\sqrt{a}+\sqrt{b})^2>(\sqrt{c}+\sqrt{d})^2$,即
$a+b+2 \sqrt{a b}>c+d+2 \sqrt{c d} \text {. }$,因为 $a+b=c+d$, 所以 $a b>c d$. \dotfill{8分}
\\于是
$(a-b)^{2}=(a+b)^{2}-4 a b<(c+d)^{2}-4 c d=(c-d)^{2} $.因此 $|a-b|<|c-d|$.\dotfill{10分}
\\综上, $\sqrt{a}+\sqrt{b}>\sqrt{c}+\sqrt{d}$ 是 $|a-b|<|c-d|$ 的充要条件.
\\ \heiti 评注:\songti 证明充要条件一定要分别证明充分性和必要性. 即:证明$ p\Leftrightarrow q $(充要条件),则要证明$ p\Rightarrow q $(充分性)和$ p\Leftarrow q $(必要性). 类似的还有:“证明……当且仅当……”,出现这样的字眼实际上也是要证明充要条件.
\subsection{\heiti 说明}
由于新高考地区没有选做题,因而选做题选自以往高考原题或是发表于《中等数学》等数学教学期刊的试题,较有权威性.本套题的22题选自\cite{ref1},23题选自2015年·全国II卷原题. 难度与高考难度持平.
\begin{thebibliography}{1}
	\bibitem{ref1} 童其林.2019年高考极坐标与参数方程考题预测[J].广东教育:高中版,2019(5):25-27+49.
\end{thebibliography}


\end{document}