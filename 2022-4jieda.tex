\documentclass[11pt]{article}
\usepackage{ctex}
\usepackage{setspace}
\usepackage{amsmath}
\usepackage{amsthm}
\usepackage{amssymb}
\usepackage{paralist}
\usepackage{enumerate}
\usepackage{tikz}
\usepackage{wrapfig}
\usepackage{blindtext}
\usepackage{pifont}
\usepackage{amsfonts}
\usepackage{geometry}
\geometry{left=2.0cm,right=2.0cm,top=2.5cm,bottom=2.5cm}

\begin{document}
	\begin{center}
		\songti \Large 2022年普通高等学校招生全国统一考试理科数学模拟4
		\\ \heiti 参考答案
	\end{center}
\section{\heiti 单项选择题}
\subsection{\heiti 答案}
\begin{center}
	\begin{tabular}{ccccccccccccc}
		\hline 
		题目 & 1 & 2 & 3 & 4 & 5 & 6 & 7 & 8 & 9 & 10 & 11 & 12 \\
		\hline
		答案 & A & D & A & B & C & D & D & C & A & C & C & B \\
		\hline
	\end{tabular}
\end{center}
\subsection{\heiti 提示}
\begin{enumerate}
	\setcounter{enumi}{5}
	\item 由题意可知应将志愿者分为三人组和两人组. 分情况讨论:当三人组中包含小明和小李时;当三人组中不包含小明和小李时.
	\item 令$ x=1 $得$ a=1 $. 然后算后面一个因式中,$ x $和$ \dfrac{1}{x} $的系数,最后相加即的常数项(与前面的因式相乘后抵消).
	\item 作代数变形$ \dfrac{a_n}{b_n}=\dfrac{a_{n-1} }{b_{n-1}} $.
	\item 单位圆内接正$ 6n $边形的周长为$ 12n \sin \dfrac{30^\circ}{n} $,外切正$ 6n $边形的周长为$ 12n \tan \dfrac{30^\circ}{n} $.
	\item $ f(x)=-\dfrac{2}{e^x+1}+\dfrac{1}{2} $. 考察$ -\dfrac{2}{e^x+1} $的值域. $ x\geqslant0 $时,$ e^x\geqslant1 $,$ -1 \leqslant -\dfrac{2}{e^x+1}<0 $. $ x<0 $时,$ 0<e^x<1 $可得$ -2<-\dfrac{2}{e^x+1}<-1 $.
	\item 考察三角形$ \triangle ABF_1 $. 设$ AF_1=x $,则$ AF_2=x-2a $. 又$ \tan \angle ABF_1 =\dfrac{3}{4} $,故$ AB=\dfrac{4}{3}x $. 则$ BF_2=\dfrac{4}{3}x-(x-2a)=\dfrac{x}{3}+2a $. 对三角形$ \triangle ABF_1 $应用勾股定理解方程. 解出$ AF_1,AF_2 $,再对$ \triangle AF_1F_2 $应用勾股定理.
	\item 注意题干“垂直向上”,因而$ A,B,C $在底面的投影是底面三角形三边的中点. 故$ \triangle ABC $投影是形成一个边长为1的等边三角形,因而$ \triangle ABC $是边长为1的等边三角形. 故$ O-ABC $是各边为1的正四面体(各个边均相等). 则$ O $到平面$ ABC $的距离是$ \dfrac{\sqrt{6}}{3} $. 平面$ ABC $到底面的距离即点$ A $到$ DE $的距离$ \sqrt{3} $. 但是球面上最短距离要减去半径.
\end{enumerate}
\section{\heiti 填空题}
\begin{enumerate}
	\setcounter{enumi}{12}
	\item \underline{\quad $ \dfrac{\ln 5}{1-\ln 5} $ \quad}.
	\\ 提示:取对数.
	\item \underline{\quad $ \dfrac{\sqrt{10}}{10} $ \quad}.
	\item \underline{\quad $ 5 $ \quad}.
	\\ 提示:样本的平均值点$ (\bar{x} ,\bar{y})  $也在回归直线上.
	\item \underline{\quad $ 6\sqrt{3}$ \quad}.
	\\ 提示:作图. 或运用正弦定理和余弦定理,像解三角形大题一样. 
\end{enumerate}
\section{\heiti 解答题}
\begin{enumerate}
	\setcounter{enumi}{16}
	\item (12分)
	\begin{enumerate}[(I)]
		\item 
		\ding{172} $2 S_{n}=n a_{n+1}$ 
		\\当 $n=1$ 时, $2 S_{1}=a_{2}$, 得 $a_{2}=2$ \dotfill{2分}
		\\当 $n \geqslant 2$ 时, $2 S_{n-1}=(n-1) a_{n}$, 得 $2 a_{n}=n a_{n-1}-(n-1) a_{n}$, 即 $(n+1) a_{n}=n a_{n+1} 
		\\ \therefore \dfrac{a_{n+1}}{a_{n}}=\dfrac{n+1}{n}$ \dotfill{4分}
		\\$\therefore a_{n}=\dfrac{n}{n-1} \times \dfrac{n-1}{n-2} \times \cdots \times \dfrac{2}{1} \times a_{1}=n$ 、\dotfill{5分}
		\\当 $n=1$ 时,$ a_1=1 $也成立, $\therefore a_{n}=n$.\dotfill{6分}
		\\ \ding{173} $2 S_{n}=a_{n+1} a_{n}$ 
		\\当 $n=1$ 时, $2 S_{1}=a_{2} a_{1}$, 得 $a_{2}=2$ \dotfill{2分}
		\\当 $n \geqslant 2$ 时, 得 $2 S_{n-1}=a_na_{n-1}$
		即$2 a_{n}=a_{n} a_{n+1}-a_{n} a_{n-1}$ 
		\\由 $a_{n}>0$, 得 $a_{n+1}-a_{n-1}=2$ \dotfill{4分}
		\\又 $\because a_{1}=1, a_{2}=2 . \therefore\left\{a_{2 n}\right\}$ 是公差为 2 , 首项为 2 的等差数
		列, $\left\{a_{2 n-1}\right\}$ 是公差为 2 , 首项为 1 的等差数列 \dotfill{5分}
		\\故 $a_{n}=n$. \dotfill{6分}
		\\ \ding{174}$a_{n}^{2}+a_{n}=2 S_{n}$ 
		\\当 $n \geqslant 2$ 时,
		$a_{n-1}^{2}+a_{n-1}=2 S_{n-1}$ \dotfill{2分}
		\\两式相减得 $a_{n}{ }^{2}+a_{n}-a_{n-1}^{2}-a_{n-1}=2 a_{n}$, 即 $\left(a_{n}+a_{n-1}\right)\left(a_{n}-a_{n-1}-1\right)=0$,\dotfill{3分}
		\\由 $a_{n}>0$, 得 $a_{n}-a_{n-1}-1=0$ \dotfill{4分}
		\\ $\therefore\left\{a_{n}\right\}$ 是公差为1,首项为1的等差数列, 
		故 $a_{n}=n$.\dotfill{6分}
		\item $b_{n}=(n+1) \cdot 2^{n}$\dotfill{7分}
		\begin{align*}
		T_{n}&=2 \times 2+3 \times 2^{2}+\cdots+\ldots+(n+1) \times 2^{n}\\
		2 T_{n}&=2 \times 2^{2}+3 \times 2^{3}+\cdots+n \times 2^{n}+(n+1) \times 2^{n+1}
		\end{align*}
		\dotfill{9分}
		\\两式相减, 得 
		\begin{align*}
			-T_{n}&=4+2^{2}+2^{3}+\ldots .+2^{n}-(n+1) \times 2^{n+1}
			=4+\dfrac{4\left(1-2^{n-1}\right)}{1-2}-(n+1) \times 2^{n+1}\\
			&=4-4+2^{n-1}-(n+1) \times 2^{n+1}=-n \cdot 2^{n+1}
		\end{align*}
		\dotfill{11分}
		\\故 $T_{n}=n \cdot 2^{n+1}$\dotfill{12分}
	\end{enumerate}
	\item (12分)
	\begin{enumerate}[(I)]
		\item 用 $X$ 表示 4 例疑似病例中化验呈阳性的人数, 则随机变量 $X \sim B\left(4, \dfrac{1}{3}\right)$ 
		\\由题意可知: $P(x=2)=C_{4}^{2} \cdot\left(\dfrac{1}{3}\right)^{2} \cdot\left(1-\dfrac{1}{3}\right)^{2}=\dfrac{8}{27}$ \dotfill{3分}
		\item 方案一: 若逐个检验, 则检验次数为 4 .\dotfill{4分}
		\\方案二: 混合一起检验, 记检验次数为 $X$, 则 $X=1,5$.
		\\$P(X=1)=\left(1-\dfrac{1}{10}\right)^{4}=\dfrac{81 \times 81}{10000}=\dfrac{6561}{10000}$  \\$P(X=5)=1-P(X=1)=\dfrac{3439}{10000}$ \dotfill{6分}
		\\$E(X)=1 \times \dfrac{6561}{10000}+5 \times \dfrac{3439}{10000}=\dfrac{23756}{10000}$ \dotfill{7分}
		\\方案三: 每组的两个样本混合在一起化验, 若结果呈阴性, 则检 测次数为 1 , 其概率为 $\left(1-\dfrac{1}{10}\right)^{2}=\dfrac{81}{100}$, 
		若结果呈阳性, 则检测次数为 3 , 其概率为 $1-\dfrac{81}{100}=\dfrac{19}{100}$ 
		\\设方案三检 测次数为随机变量 $Y$, 则$Y=2,4,6$ \\$P(Y=2)=\dfrac{81}{100} \times \dfrac{81}{100}=\dfrac{81 \times 81}{10000}=\dfrac{6561}{10000} $
		\\$P(Y=4)=\dfrac{81}{100} \times \dfrac{19}{100} \times 2=\dfrac{2 \times 81 \times 19}{10000}=\dfrac{3078}{10000}
		$
		\\ $P(Y=6)=\dfrac{19}{100} \times \dfrac{19}{100}=\dfrac{361}{10000}$ \dotfill{10分}
		\\则$E(Y)=2 \times \dfrac{81 \times 81}{10000}+4 \times \dfrac{2 \times 81 \times 19}{10000} \times \dfrac{19 \times 19}{10000}=\dfrac{27600}{10000} $ \dotfill{11分} 
		\\由 $E(X)<E(Y)<4$, 知方案二最优.\dotfill{12分}
	\end{enumerate}
	\item (12分)
	\begin{enumerate}[(I)]
		\item 如图, 取 $A B$ 的中点 $M$, 连结 $D M, D B,$
		\\$ \because C D=\dfrac{1}{2} A B, \therefore C D=M B, \because C D / / M B, \therefore$ 四 边形 $B C D M$ 为平行四边形, $\therefore D M=B C$
		\\$ \because$ 四边形 $A B C D$ 是等腰梯形, $A B / / C D, \therefore D M=B C=A D$, 
		\\又 $A D=C D=\dfrac{1}{2} A B=A M, \therefore \triangle A M D$ 为等边三角形, $\therefore \angle D A M=\angle D M A=60^{\circ}$,
		\\$\therefore$ 在等腰 $\triangle M B D$ 中, $\angle M B D=30^{\circ}, \therefore$ 在 $\triangle A D B$ 中, $\angle A D B=90^{\circ}$, \\不妨设
		$2 P D=2 A D=2 C D=A B=P B=2$, 则 $B D=\sqrt{3}$, 在 $\triangle P B D$ 中, $B D=\sqrt{3}, P D=1, P B=2$,
		$\therefore P D^{2}+B D^{2}=P B^{2}, \therefore P D \perp B D$,\dotfill{3分}
		\\又 $P D \perp A D, A D \subset$ 羊面 $A B C D, B D \subset$ 平面 $A B C D, A D \cap B D=D$,
		$\therefore P D \perp$ 平面 $A B C D$, 
		\\又 $P D \subset$ 平面 $P A D, \therefore$ 平面 $P A D \perp$ 平面 $A B C D$.\dotfill{5分}
		\item $\because P D \perp A D, P D \perp B D, A D \perp B D, \therefore$ 以 $A D, B D, P D$ 分别为 $x$ 轴, $y$ 轴, $z$ 轴建立空间直角坐标系, 如图.\dotfill{6分}
		\\设 $P D=1, \because$ 平面 $P D E$ 把四棱雉 $P-A B C D$ 分成体积相等的两部分, 三棱雉 $P-A D E$ 的体积等于四棱 锥 $P-B C D $的体积
		\\$\therefore \dfrac{1}{3} S_{\triangle A D C} \times P D=\dfrac{1}{3} S_{\text {梯形 } D C B E} \times P D , \therefore S_{\triangle A D E}=S_{\text {梯形 } D C B E}$
		\\设梯形 $A B C D$ 的高为 $h, A E=x$ ,则 $\dfrac{1}{2} x h=\dfrac{1}{2} \cdot (2-x+1) h$, 解得 $x=\dfrac{3}{2}$, \dotfill{7分}
		\\故 $D(0,0,0), A(1,0,0), B(0, \sqrt{3}, 0), P(0,0,1)$,$E\left(\dfrac{1}{4}, \dfrac{3 \sqrt{3}}{4}, 0\right), C\left(-\dfrac{1}{2}, \dfrac{\sqrt{3}}{2}, 0\right) ,$
		\\$\overrightarrow{P N}=\left(-\dfrac{1}{2}, \dfrac{\sqrt{3}}{2},-1\right), \overrightarrow{P E}=\left(\dfrac{1}{4}, \dfrac{3 \sqrt{3}}{4},-1\right)$,
		\\$ \because $平面$ PAD \bigcap $平面$ ABCD =AD$,$ BD\subset  $平面$ ABCD $,$\therefore y$ 轴 $\perp$ 平面 $P A D, \therefore$ 平面 $P A D$ 的一个法向量为 $\vec{n}=(0,1,0)$\dotfill{8分}
		\\设平面 $P C E$ 的一个法向量为 $\vec{m}=(x, y, z)$, 则 $\left\{\begin{array}{l}\vec{m} \cdot \overrightarrow{P C}=0 \\ \vec{m} \cdot \overrightarrow{P E}=0\end{array}\right.$ ,
		\\即 $\left\{\begin{array}{l}-\dfrac{1}{2} x+\frac{\sqrt{3}}{2} y-z=0 \\ \dfrac{1}{4} x+\dfrac{3 \sqrt{3}}{4} y-z=0\end{array}\right.$
		\\取 $x=-\sqrt{3}$, 则 $y=3, z=2 \sqrt{3},$$ \therefore \vec{m}=(-\sqrt{3}, 3,2 \sqrt{3}), $\dotfill{10分}
		\\$\therefore \cos \left \langle \vec{m},\vec{n}   \right \rangle =\dfrac{\vec{m} \cdot \vec{n}}{|\vec{m}||\vec{n}|}=\dfrac{3}{2 \sqrt{6}}=\dfrac{\sqrt{6}}{4}, $\dotfill{11分}
		\\$\therefore$ 平面 $P A D$
		与平面 $P C E$ 所成锐二面角的余弦值为 $\dfrac{\sqrt{6}}{4}$.\dotfill{12分}
	\end{enumerate}
	\item (12分)
	\begin{enumerate}[(I)]
		\item 过 $\mathrm{A}, B$ 分别向 $N D$ 作垂线, 垂足为 $A^{\prime}, B^{\prime}$, 设 $A B$ 中点为 $P$, 过 $P$ 向 $N D$ 作 垂线垂足为 $P^{\prime}$, 
		\\则 $\left|P P^{\prime}\right|=\dfrac{1}{2}\left(\left|A A^{\prime}\right|+\left|B B^{\prime}\right|\right)=\dfrac{1}{2}|A B|$.
		\\又 $\because|A B|=|B C| \therefore\left|P P^{\prime}\right|=\dfrac{1}{3}|P C|$. $ \left | P'C \right | =\sqrt{\left | PC \right |^2-\left | PP' \right |^2  }=\dfrac{2\sqrt{2} }{3}\left | PC \right |   $.
		\\在 Rt $\Delta P P^{\prime} C$ 中, $\tan \angle P^{\prime} P C=2 \sqrt{2}$. 故直线 $l$ 的斜率为 $2 \sqrt{2}$.\dotfill{4分}
		\item $\because$ 正方形边长为 $1, \therefore \left | FD \right | =p=1$, 抛物线方程为 $y^{2}=2 x, \therefore M\left(\dfrac{1}{2}, 1\right)$, \dotfill{6分}
		\\设直线$ AB $的方程为
		$y=k\left(x-\dfrac{1}{2}\right), A\left(x_{1}, y_{1}\right), B\left(x_{2}, y_{2}\right), $
		\\$\because N D$ 方程为 $x=-\dfrac{1}{2}$, 得 $C\left(-\dfrac{1}{2},-k\right), \therefore k_{3}=\dfrac{1+k}{\dfrac{1}{2}+\dfrac{1}{2}}=1+k$ \dotfill{7分}
		\\由 $\left\{\begin{array}{l}y=k\left(x-\dfrac{1}{2}\right), \\ y^{2}=2 x\end{array}\right.$,得$4 k^{2} x^{2}-\left(4 k^{2}+8\right) x+k^{2}=0,$
		\\$ \Delta=\left(4 k^{2}+8\right)^{2}-4 \cdot 4 k^{2} \cdot k=64 k^{2}+64>0 . $
		\\$x_{1}+x_{2}=\dfrac{4 k^{2}+8}{4 k^{2}}=\dfrac{k^{2}+2}{k^{2}}, x_{1} x_{2}=\dfrac{k^{2}}{4 k^{2}}=\dfrac{1}{4}, 
		$\dotfill{8分}
		\\$ k_{1}=\dfrac{y_{1}-1}{x_{1}-\dfrac{1}{2}}, k_{2}=\dfrac{y_{2}-1}{x_{2}-\dfrac{1}{2}}, $
		\\
		\begin{align*}
			k_{1}+k_{2}&=\dfrac{y_{1}-1}{x_{1}-\dfrac{1}{2}}+\dfrac{y_{2}-1}{x_{2}-\dfrac{1}{2}}
			=\dfrac{\left(y_{1}-1\right)\left(x_{2}-\dfrac{1}{2}\right)+\left(x_{1}-\dfrac{1}{2}\right)\left(y_{2}-1\right)}{\left(x_{1}-\dfrac{1}{2}\right)\left(x_{2}-\dfrac{1}{2}\right)}
			\\
			&=\frac{\left[k\left(x_{1}-\dfrac{1}{2}\right)-1\right]+\left[k\left(x_{2}-\dfrac{1}{2}\right)-1\right]\left(x_{1}-\dfrac{1}{2}\right)}{x_{1} x_{2}-\dfrac{1}{2}\left(x_{1}+x_{2}\right)+\dfrac{1}{4}}
			\\ &=\dfrac{2 k x_{1} x_{2}-(k+1)\left(x_{1}+x_{2}\right)+\dfrac{k}{2}+1}{x_{1} x_{2}-\dfrac{1}{2}\left(x_{1}+x_{2}\right)+\dfrac{1}{4}}
			=\dfrac{\dfrac{k}{2}-\dfrac{(k^2+2)(k+1)}{k^2}+\dfrac{k}{2}+1   }{\dfrac{1}{4}-\dfrac{k^2+2}{2k^2}+\dfrac{1}{4}
			}
			\\ &=-k^2(k+1)+(k^2+2)(k+1)=2(k+1)=2k_3
		\end{align*}
		即存在常数 $\lambda=2$, 使得 $k_{1}+k_{2}=2 k_{3}$ 成立.\dotfill{12分}
	\end{enumerate}
	\item (12分)
	\begin{enumerate}[(I)]
		\item $f(x)$ 的定义域为 $(0,+\infty)$ .
		$f^{\prime}(x)=-\dfrac{a}{x^{2}}-\dfrac{1}{2}+\dfrac{1}{x}=-\dfrac{x^{2}-2 x+2 a}{2 x^{2}} .$\dotfill{1分}
		\\对$ y=x^2-2x+2a,\Delta =4-8a. $
		\\若 $\Delta \geq 0$ ,即 $a \geq \dfrac{1}{2}$ ,则 $x^{2}-2 x+2 a \geq 0$ 恒成立,
		即 $f^{\prime}(x) \leq 0 , f(x)$ 在 $(0,+\infty)$ 上单调递减.\dotfill{2分}
		\\否则,知$ x^2-2x+2a=0 $的两根为$ x_1=1-\sqrt{1-2a} ,x_2=1+\sqrt{1-2a}  $.
		\begin{enumerate}
			\item 若 $1-\sqrt{1-2 a}>0$ ,即 $0<a<\dfrac{1}{2}$ ,
			\\当 $x \in(0,1-\sqrt{1-2 a})$ 或 $x \in(1+\sqrt{1-2 a},+\infty)$ 时, $f^{\prime}(x)<0 ,f(x)$ 单调递减.
			\\当 $x \in(1-\sqrt{1-2 a}, 1+\sqrt{1-2 a})$ 时, $f^{\prime}(x)>0 ,f(x)$ 单调递增;\dotfill{3分}
			\item 若 $1-\sqrt{1-2 a} \leq 0$ , 即 $a \leq 0$ ,
			\\当 $x \in(1+\sqrt{1-2 a},+\infty)$ 时, $f'(x)<0 , f(x)$ 单调递减.
			\\ 当 $x \in(0,1+\sqrt{1-2 a})$ 时, $f^{\prime}(x)>0 , f(x)$ 单调递增.\dotfill{4分}
		\end{enumerate}
		综上,$a \geqslant \dfrac{1}{2} , f(x)$ 在 $(0,+\infty)$ 上单调递减.
		\\$0<a<\dfrac{1}{2} , f(x)$ 在 $(0,1-\sqrt{1-2 a}) $或$(1+\sqrt{1-2 a},+\infty)$ 上单调递减,在 $(1-\sqrt{1-2 a}, 1+\sqrt{1-2 a})$ 上单调递增.
		\\$a \leqslant 0 , f(x)$ 在 $(1+\sqrt{1-2 a},+\infty)$ 上单调递减,在 $(0,1+\sqrt{1-2 a})$ 上单调递增.\dotfill{6分}
		\item 
		由 (I) 可知 $0<a<\dfrac{1}{2}$ ,且 $x_{1}+x_{2}=2 , x_{1} x_{2}=2 a$ ,\dotfill{7分}
		\\则 $f\left(x_{1}\right)+f\left(x_{2}\right)=a\left(\dfrac{1}{x_{1}}+\dfrac{1}{x_{2}}\right)-\dfrac{1}{2}\left(x_{1}+x_{2}\right)+\ln x_{1}+\ln x_{2}=a \dfrac{x_{1}+x_{2}}{x_{1} x_{2}}-\dfrac{1}{2}\left(x_{1}+x_{2}\right)+\ln \left(x_{1} x_{2}\right)=\ln (2 a)$ ,\dotfill{8分}
		\\欲证不等式即 $\ln (2 a)<\mathrm{e}^{2 a}-2$ ,
		设 $t=2 a$ ,则 $0<t<1$ , 即证 $\mathrm{e}^{t}-\ln t-2>0(0<t<1)$ ,
		\\设 $h(t)=\mathrm{e}^{t}-\ln t-2$ ,则 $h^{\prime}(t)=\mathrm{e}^{t}-\dfrac{1}{t}$ ,
		\\$ h''(t)=e^t+\dfrac{1}{t}>0 $在$ (0,1) $上恒成立,故$h^{\prime}(t)$ 在 $(0,1)$ 上单调递增.
		\\因为 $h^{\prime}\left(\dfrac{1}{3}\right)=\mathrm{e} ^{\frac{1}{3}}-3<0 , h^{\prime}(1)=\mathrm{e}-1>0$ ,
		所以 $h^{\prime}(t)=0$ 在 $(0,1)$ 内有唯一根 $t_{0}$ , 即 $\mathrm{e}^{t_{0}}=\dfrac{1}{t_{0}}$ ,\dotfill{10分}
		\\当 $t \in\left(0, t_{0}\right)$ 时, $h^{\prime}(t)<0 , h(t)$ 单调递减,当 $t \in\left(t_{0}, 1\right)$ 时, $h^{\prime}(t)>0 , h(t)$ 单调递增,
		\\所以 $h(t)_{\min }=h\left(t_{0}\right)=\mathrm{e}^{t_{0}}-\ln t_{0}-2=\dfrac{1}{t_{0}}-\ln \dfrac{1}{\mathrm{e}^{t_{0}}}-2=\dfrac{1}{t_{0}}+t_{0}-2>0$ ,
		$h(t)>0(0<t<1)$ , 故原命题得证.\dotfill{12分}
	\end{enumerate}
	\item (10分)
	\begin{enumerate}[(I)]
		\item $ l $的直角坐标方程为 $y=\sqrt{3} x$ ,化为极坐标方程为 $\theta=\dfrac{\pi}{3}(\rho \in R)$. 
		\\将圆 $C$ 的参数方程变形为 $\left\{\begin{array}{l}x-a=\cos \alpha, \\ y=\sin \alpha,\end{array}\right.$ 平方相加得 $(x-a)^{2}+y^{2}=1$ , 
		\\化为极坐标方程为 $\rho^{2}-2 a \rho \cos \theta+a^{2}-1=0$.\dotfill{4分}
		\item 将 $\theta=\dfrac{\pi}{3}$ 代入圆 $C$ 的极坐标方程得 $\rho^{2}-a \rho+a^{2}-1=0$.
		\\设 $\left|\rho_{1}\right|=|O P|$ , $\left|\rho_{2}\right|=|O Q|$ ,则 $\rho_{1}+\rho_{2}=a , \rho_{1} \rho_{2}=a^{2}-1$ ,
		\\$\Delta=a^{2}-4\left(a^{2}-1\right)>0$ ,解得 $0 \leq a^{2}<\dfrac{4}{3}$.
		\\所以 $|O P|^{2}+|O Q|^{2}=\rho_{1}^{2}+\rho_{2}^{2}=\left(\rho_{1}+\rho_{2}\right)^{2}-2 \rho_{1} \rho_{2}=a^{2}-2\left(a^{2}-1\right)=2-a^{2}$. 
		\\所以 $|O P|^{2}+|O Q|^{2}$ 的取值范围是 $\left(\dfrac{2}{3}, 2\right]$.\dotfill{10分}
	\end{enumerate}
	\item (10分)
	\begin{enumerate}[(I)]
		\item $f\left(\dfrac{1}{2}\right)+f(-1) \geq 8 $ ,即 $|a+1|+|a-2|+3 \geq 8 $ ,亦即 $|a+1|+|a-2| \geq 5 $,
		\\等价于不等式组 $\left\{\begin{array}{l}a \leq-1 \\ -a-1-a+2 \geq 5\end{array} ,\right.$ 或 $\left\{\begin{array}{l}-1<a \leq 2 \\ a+1-a+2 \geq 5\end{array} ,\right.$ 或 $\left\{\begin{array}{l}a>2 \\ a+1+a-2 \geq 5\end{array}\right.$ 
		\\解得 $a \leq-2$ 或 $a \geq 3 $,
		故实数 $a$ 的取值范围是 $(-\infty,-2] \bigcup [3,+\infty)$.\dotfill{5分}
		\item 
		原命题
		等价于 $f(x)_{\min }<\left(b+\dfrac{1}{b-1}+1\right)_{\min } .$
		\\因为 $f(x)=|2 x+a|+|2 x-1| \geqslant |(2x+a)-(2x-1)|=|a+1|$, 当且仅当$ (2x+a)(2x-1)\leqslant 0 $等号成立,所以 $f(x)_{\min }=|a+1| .$\dotfill{7分}
		\\又 $b+\dfrac{1}{b-1}+1=b-1+\dfrac{1}{b-1}+2 \geqslant 4 ,b \in(1,+\infty)$ ,当且仅当 $b=2$ 时取等号,所以 $\left(b+\dfrac{1}{b-1}+1\right)_{\min }=4 .$\dotfill{9分}
		\\由 $|a+1|<4 ,$ 解得 $-5<a<3 $.
		\\故所求实数 $a$ 的取值范围是 $(-5,3) .$\dotfill{10分}
	\end{enumerate}
\end{enumerate}
\section{\heiti 说明与注意}
\subsection{\heiti 说明}
	本题改变自某新高考模拟卷. 其中第8、9题是北京市高考原题,第13题是自创原题,保证尽量覆盖原卷精神和知识点比例. 中档题居多,但整体难度不算高. 
\subsection{\heiti 注意}
	第10题的取整函数又来了,如果你这次还是不会写,从两次的考察中总结其通用方法(提示:考虑定义域和值域,并尝试画出取整函数的图像).
\end{document}