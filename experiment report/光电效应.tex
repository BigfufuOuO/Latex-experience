\documentclass[11pt]{article}
\usepackage{ctex}
\usepackage{setspace}
\usepackage{amsmath}
\usepackage{amsthm}
\usepackage{amssymb}
\usepackage{paralist}
\usepackage{enumerate}
\usepackage{tikz}
\usepackage{wrapfig}
\usepackage{blindtext}
\usepackage{amsfonts}
\usepackage{geometry}
\usepackage{graphics}
\usepackage{theorem}
\usepackage{booktabs}
\usepackage{wrapfig}
\usepackage{multirow}
\geometry{left=2.0cm,right=2.0cm,top=2.5cm,bottom=2.5cm}

\begin{document}
	\begin{center}
		\Large \heiti 光电效应测普朗克常数实验报告
	\end{center}
\section{实验目的}
测量普朗克常数 $ h $
\section{实验原理}
单色光照射在光电管的阴极上有电子发射出来的现象叫光电效应, 出射的电子称之为光电子,形成的电流称之为光电流。 光电流很弱。 加载在光电管中阳极与阴极之间电压为正值时,随着电压的增大光电流迅速增大,电压增大到一定值后,光电流趋于饱和。 加载在阳极与阴极之间电压为负值时, 随着电压数值逐渐变大,光电流变弱, 负电压数值增大到 $ U_0 $ 值时,光电流变为零。 把电压 $ U_0 $ 称之为遏止电压。本实验要求测量 5 种不同单色光分别照射下, 光电流的遏止电压值。 本实验还需测量和验证饱和光电流与光强之间的关系,是否满足线性正比关系。
\section{实验仪器}
使用ZKY-GD-4 智能光电效应(普朗克常数)实验仪。如图\ref{fig:1}。
\begin{figure}[htbp]
	\centering
	\includegraphics[width=0.7\linewidth]{D:/STUDY/物理/实验方案与报告/光电效应(不交)/光电效应仪器}
	\caption{仪器结构示意图}
	\label{fig:1}
\end{figure}
\section{实验步骤及数据处理}
\subsection{零电流法、补偿法分别测遏止电压}
固定一种直径大小光阑的情况下, 分别测量 5 种不同单色光照射下,光电流的遏止电压。

\kaishu a)测量数据记录如表1。

b)用最小二乘法计算普朗克常数 $ h $ 大小,以及与公认值 $ h_0 $之间的相对误差

c)计算此光电管阴极材料, 产生光电效应的单色照射光的波长红限,以及光电子从材料表面逸出的功大小。\songti
\begin{table}[htbp]\small
	\caption{不同单色光照射下的光电流的遏止电压}
	\centering
	\begin{tabular}{cccccc}
		\multicolumn{6}{r}{光阑孔$ \Phi=4  $mm}                       \\
		\toprule
		波长$ \lambda_i $(nm)   & 365   & 404.7 & 435.8 & 546.1 & 577   \\
		频率$ \nu _i $($ \times 10^{14} $Hz)   & 8.214 & 7.408 & 6.879 & 5.49  & 5.196 \\
		\midrule
		零电流法$ U_0 $(V) & 1.788 & 1.508 & 1.176 & 0.590 & 0.476 \\
		\midrule
		补偿法$ U_0 $(V)  & 1.798 & 1.508 & 1.178 & 0.588 & 0.474 \\
		\bottomrule
		\multicolumn{6}{r}{主机箱编号: 1823  光电管外壳编号:1823} \\
	\end{tabular}
\end{table}

数据处理和分析:对表 1 中的单色光频率与遏止电压(正值)之间,在直角坐标纸上进行画图、描点,再进行(最小二乘)线性回归拟合分析,做出拟合直线。 写出拟合直线方程。(见图\ref{fig:2}和\ref{fig:3})
% TODO: \usepackage{graphicx} required
\begin{figure}[htbp]
	\centering
	\includegraphics[width=0.6\linewidth]{D:/STUDY/物理/实验方案与报告/光电效应(不交)/零电流法}
	\caption{零电流法线性拟合}
	\label{fig:2}
\end{figure}
% TODO: \usepackage{graphicx} required
\begin{figure}[htbp]
	\centering
	\includegraphics[width=0.6\linewidth]{../补偿法}
	\caption{补偿法线性拟合}
	\label{fig:3}
\end{figure}

\heiti 零电流法:\songti 线性拟合得拟合直线方程为$ y=(0.44428\pm 0.01664)x-(1.81429\pm 0.11206) $.
根据上述拟合直线方程,计算:

1.普朗克常数值为$ h=ek=(7.112\pm 0.267)\times 10^{-34}J\cdot S $.

2.与公认值比较计算相对误差$ E=\dfrac{\left| h-h_0\right| }{h}=0.07416 \pm 0.04023$,得相对误差在$ 3.39\%\sim 11.44\% $.

3.计算单色入射光逸出功为$ W_0=-be=(1.814\pm 0.112) $eV.

4.计算单色入射光红限为$ \nu=(4.080\pm 0.099)\times 10^{14} $Hz.

\heiti 补偿法:\songti 线性拟合得拟合直线方程为$ y=(0.44756\pm 0.0158)x-(1.86146\pm 0.10643) $.
根据上述拟合直线方程,计算:

1.普朗克常数值为$ h=ek=(7.169\pm 0.253)\times 10^{-34}J\cdot S $.

2.与公认值比较计算相对误差$ E=\dfrac{\left| h-h_0\right| }{h}=0.08209\pm 0.03820$,得对误差在$ 4.38\%\sim 12.01\% $.

3.计算单色入射光逸出功为$ W_0=-be=(1.861\pm 0.106) $eV.

4.计算单色入射光红限为$ \nu=(4.156\pm 0.091)\times 10^{14} $Hz.
\subsection{饱和光电流与光强之间的变化关系}
测量饱和光电流与光强的关系。步骤如下:

\kaishu a) 其一种情况是, 选择一种单色光,固定光电管阴阳极电压(在饱和区),改变不同的光阑(直径)大小,来改变光强.

b) 另一种情况是, 选择一种单色光,固定光电管阴阳极电压(在饱和区),改变光电管与汞灯光源的距离,来改变光强.

c) 二种测量内容,分别列表, 画图。验证饱和光电流与光强, 成正比关系。\songti

得到数据如表2和3。由表格数据画图如图\ref{fig:4}和\ref{fig:5}。其中光照强度$ P $与光阑直径的关系为$ P\propto \Phi^2 $,距离$ L $关系为$ P\propto \dfrac{1}{L^2}  $,故图\ref{fig:4}的横坐标为$ \Phi^2 $,图\ref{fig:5}的横坐标为$ \dfrac{1}{L^2} $。可以看出,在实验允许的误差范围内,饱和电流与光照强度成正比。
% Please add the following required packages to your document preamble:
% \usepackage{multirow}
\begin{table}[htbp]\small
	\caption{$ I_M-P $关系(改变光阑)}
	\centering
	\begin{tabular}{c|cccc}
		\multicolumn{5}{r}{$ U $=20V, $ L $=400mm}                    \\
		\toprule
		\multirow{2}{*}{435.8nm} & $ \Phi $(nm) & 2   & 4     & 8     \\
		& $ I $($ 10^{-10} $A)   & 8.2 & 31.0 & 124.3 \\
		\midrule
		\multirow{2}{*}{546.1nm} & $ \Phi $(nm) & 2   & 4     & 8     \\
		& $ I $($ 10^{-10} $A)    & 1.0 & 3.7   & 13.9 \\
		\bottomrule
	\end{tabular}
\end{table}
% Please add the following required packages to your document preamble:
% \usepackage{multirow}
\begin{table}[htbp]
	\caption{$ I_M-P $关系(改变距离)}
	\centering
	\begin{tabular}{c|cccccc}
		\multicolumn{7}{r}{$ U $=20V, $ \Phi $=4mm}                                  \\
		\toprule
		\multirow{2}{*}{435.8nm} & $ L $(mm) & 300  & 320 & 340  & 360  & 380  \\
		& $ I $($ 10^{-10} $A)  & 66.1 & 55.0        & 46.9 & 39.6 & 33.8 \\
		\midrule
		\multirow{2}{*}{546.1nm} & $ L $(mm) & 300  & 320 & 340  & 360  & 380  \\
		& $ I $($ 10^{-10} $A)  & 7.6  & 6.3 & 5.4  & 4.7  & 4.1 \\
		\bottomrule
	\end{tabular}
\end{table}
% TODO: \usepackage{graphicx} required
\begin{figure}[htbp]
	\centering
	\includegraphics[width=0.7\linewidth]{../改变光阑}
	\caption{$ I_M-P $关系图(改变光阑)}
	\label{fig:4}
\end{figure}
% TODO: \usepackage{graphicx} required
\begin{figure}
	\centering
	\includegraphics[width=0.7\linewidth]{../改变距离}
	\caption{$ I_M-P $关系图(改变距离)}
	\label{fig:5}
\end{figure}
\end{document}