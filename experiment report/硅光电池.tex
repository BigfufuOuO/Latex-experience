\documentclass[11pt]{article}
\usepackage{ctex}
\usepackage{setspace}
\usepackage{amsmath}
\usepackage{amsthm}
\usepackage{amssymb}
\usepackage{paralist}
\usepackage{enumerate}
\usepackage{tikz}
\usepackage{wrapfig}
\usepackage{blindtext}
\usepackage{amsfonts}
\usepackage{geometry}
\usepackage{graphics}
\usepackage{theorem}
\usepackage{booktabs}
\usepackage{wrapfig}
\usepackage{multirow}
\geometry{left=2.0cm,right=2.0cm,top=2.5cm,bottom=2.5cm}

\begin{document}
	\begin{center}
		\Large \heiti 硅光电池的实验报告
	\end{center}

\section{实验目的}
1.了解硅光电池工作原理.

2.掌握硅光电池的工作特性.
\section{实验原理}
硅光电池是根据光伏效应而制成的将光能转换成电能的一种器件,它的基本结构就是一个 P-N 结。 当 P 型和 N 型半导体材料结合时, P
型材料中的空穴向 N 型材料这边扩散,使P 型区出现负电荷, N 型区带正电荷,形成一个势垒。由此而产生的内电场将阻止扩散运
动的继续进行,当两者达到平衡时,在 PN 结两侧形成一个耗尽区。势垒通过外加电场的作用下使PN结形成单向导电性,电流方向是从 P 指向 N,即P为正,N为负。

当硅光电池 PN 结处于零偏或反偏时,在它们的结合面耗尽区存在一内电场,当有光照时, 电池对光
子的本征吸收和非本征吸收都产生光生载流子,当在 PN 结两端加负载时就有一光生电流流过负载。

\heiti 伏安特性:\songti
在一定光照下,在光电池两端加一个负载就会有电流流过,当负载很大时,电流较小而电压较大;当负载很小时,电流较大而电压较小。在无偏压工作状态下,光电流随负载变化很大。

\heiti 照度特性:\songti 当没有光照时,硅光电池等效于普通的二极管,对于外加正向电压, $ I $ 随$ V $ 指数增长,称正向电流;当外加电压为反向时,在
反向击穿电压之内,反向饱和电流基本是个常数。当有光照时,入射光子把处于价带的束缚电子激发到导带,激发出的电子空穴对在内电场作用下分别
飘逸到 N 型区和 P 型区,当在 PN 结两端加负载时就有光生电流流过负载,流过 PN 结两端的电流:
$$ I=I_{ph}-I_0\left [ \exp \left ( \dfrac{qU}{k_BT}  \right ) -1 \right ]  $$
式中$ I_{ph}  $是与入射光的强度成正比的光生电流,其比例系数与负载电阻的大小及硅光电池的结构特性有关.
\\ 当硅光电池在短路状态时($ U=0 $),短路电流为:$ I_{SC}=I_{ph} $;当硅光电池在开路状态时($ I=0 $),开路电压为$ U_{OC}=\dfrac{k_BT}{q}\ln \left( \dfrac{I_{SC}}{I_0} +1\right)  $. 即短路电流$ I_{SC} $和光照强度 $ L  $成正比,开路电压$ U_{OC }$ 与光照强度 $ L $ 的对数成正比。在线性测量中,光电池通常以电流形式使用,故$  I_{SC}  $与 $ L $ 呈线性关系,是光电池的重要光照特性。实际使用时都接有负载电阻 $ R_L $,输出电流 $ I_L $ 随 $ L  $的增加而非线性缓慢地增加,并且随$  R_L $的增大线性范围也越来越小。因此,在要求输出的$  I_L  $与$  L $ 呈线性关系时, $ R_L $在条件许可的情况下越小越好,并限制在光照范围内使用。故本实验的短路电流测量时,所使用的的是$ 50\Omega $的电阻。光照强度在40—250 lx范围内。

\heiti 输出特性:\songti 硅光电池负载$  R_L $上的电压降$ U $ 和通过$  R_L $的电流之积称为硅光电池的输出功率$  P  $。在一定的照度下,不同$  R_L $有不同的输出功率$  P  $,输出功率达到最大值$  P_m  $时的负载电阻$  R_m  $称为最佳负载电阻。此时能量转换效率最高,且$ R_m $随光强而变化。 当$ R_m=R_L $时, $ P_m=U_mI_m $,式中$ U_m  $和 $ I_m $ 分别是最佳工作电压和最佳工作电流,为最大输出功率。填充因子定义为$ FF=\dfrac{P_m}{U_{OC}I_{SC}} $,$ FF $ 越大则输出功率越高,说明硅光电池对光的利用率越高。
\section{实验仪器}
硅光电池、 数字万用表、 毫安表、 电阻箱、 溴钨灯、 直流稳压电源、光学导轨及支座、开关、导线若干。
\section{实验步骤及测量记录}
\subsection{硅光电池暗伏安特性测量}
在没有光照(全黑)下,按图\ref{fig:4}所示连接电路,测量20组电流$ I $和电压$ U $。由于试验所使用的电流表满偏为100mA,则间隔5mA读取一组数据。实验中所用电流表量程$ I=20\text{mA} $,电压表量程$ U=2\text{V} $.
经处理后得到表1,将其用$ y=ae^x+b $曲线拟合,得到图\ref{fig:11}.

\begin{table}[htbp]\small
	\label{table1}
	\caption{硅光电池暗伏安特性}
	\begin{tabular}{cccccccccccccc}
%去除原始数据 
	\end{tabular}
\end{table}

\begin{wrapfigure}{r}{0.6\linewidth}
	\centering
	\includegraphics[width=\linewidth]{../实验1图1}
	\caption{硅光电池暗伏安特性曲线}
	\label{fig:11}
\end{wrapfigure}

拟合得$ I=0.708\exp\left( \dfrac{U}{0.377}\right)-3.36  $,表明电流$ I $随电压$ U $呈指数增长。
\subsection{硅光电池输出特性测量}
不加偏压,用溴钨灯照射硅光电池,测量不同$  L  $、不同$  R $下硅电池的工作电压$ U $。电路按如图\ref{fig:5}所示连接。溴钨灯到硅光电池的距离($ d $)为50 cm时,光照强度($ L $)为40 lx. 根据$ L \propto \dfrac{1}{d^2}  $,即可计算出其他距离的光照强度。实验中所使用的的电压表量程为$ U=2\text{V} $.整理后得到的数据如表2和. 分别画出$ I-U $图像和$ P-R $图像如图\ref{fig:21}和图\ref{fig:22}. 由图可知,不同光照下对应的最大功率$ P_m $和最加负载电阻$ R_m $的关系如表3.
\begin{table}[htbp]\small
	\label{table3}
	\caption{硅光电池输出特性}
	\begin{tabular}{ccccccccccccc}
%去除原始数据
	\end{tabular}
\end{table}
% TODO: \usepackage{graphicx} required
\begin{figure}[htbp]
	\centering
	\includegraphics[width=\linewidth]{../实验2图1}
	\caption{硅光电池输出特性的$ I-U $曲线}
	\label{fig:21}
\end{figure}
% TODO: \usepackage{graphicx} required
\begin{figure}[htbp]
	\centering
	\includegraphics[width=0.7\linewidth]{../实验2图2}
	\caption{硅光电池输出特性的$ P-R $曲线. \kaishu 图中R的刻度是以10位底的对数刻度}
	\label{fig:22}
\end{figure}

\begin{table}[htbp]
	\centering
	\caption{不同光照下对应的最大功率$ P_m $和最加负载电阻$ R_m $的关系}
	\begin{tabular}{cccc}
%去除原始数据
	\end{tabular}
\end{table}
% TODO: \usepackage{graphicx} required
\begin{figure}[htbp]
	\centering
	\begin{minipage}{0.5\linewidth}
		\centering
		\includegraphics[width=0.5\linewidth]{../实验1电路}
		\caption{硅光电池暗伏安特性测量的电路}
		\label{fig:4}
	\end{minipage}
	\begin{minipage}{0.4\linewidth}
		\centering
		\includegraphics[width=\linewidth]{../实验2电路}
		\caption{硅光电池输出特性测量的电路}
		\label{fig:5}
	\end{minipage}
\end{figure}

\subsection{硅光电池开路电压$ U_{OC} $ 、短路电流 $ I_{SC}  $ 与光照$  L $ 特性测量}
电路连接分别如图\ref{fig:31}和如\ref{fig:32}所示,测量不同光照下硅光电池的开路电压$ U_{OC}  $、短路电流$  I_{SC}  $. 实验中测量开路电压时,电压表量程为$ U=2\text{V} $. 测量短路电流时,使用的电压表量程为$ U=200\text{mV} $,所接的电阻为$ R=50 $Ω.
\begin{figure}[htbp]
	\centering
	\begin{minipage}{0.4\linewidth}
		\centering
		\includegraphics[width=\linewidth]{../实验3电路1}
		\caption{测量硅光电池开路电压的电路}
		\label{fig:31}
	\end{minipage}
	\begin{minipage}{0.4\linewidth}
		\centering
		\includegraphics[width=\linewidth]{../实验3电路2}
		\caption{测量硅光电池开路电流的电路}
		\label{fig:32}
	\end{minipage}
\end{figure}

测量得到的数据如表4.
\begin{table}[htbp]\small
	\centering
	\caption{硅光电池的开路电压和短路电流}
	\begin{tabular}{cccccccc}
%去除原始数据
	\end{tabular}
\end{table}

由上表画出$ U_{OC}-L $和$ I_{SC}-L $曲线如图\ref{fig:41}和图\ref{fig:42}所示.
% TODO: \usepackage{graphicx} required
\begin{figure}[htbp]
	\centering
	\includegraphics[width=0.75\linewidth]{../实验4图1}
	\caption{硅光电池的$ U_{OC}-L $曲线}
	\label{fig:41}
\end{figure}
% TODO: \usepackage{graphicx} required
\begin{figure}[htbp]
	\centering
	\includegraphics[width=0.75\linewidth]{../实验4图2}
	\caption{硅光电池的$ I_{SC}-L $曲线}
	\label{fig:42}
\end{figure}

对$ U_{OC}-L $对数拟合,即使用$ y=a\ln (bx) $拟合,得$ a=0.0261,b=98561.462 $,故$$ U_{OC}=0.0261\cdot \ln(98561.462 L) $$
对$ I_{SC}-L $线性拟合,即$ y=a+bx $拟合,得$ a=0.01575\pm 0.01007 $,$ b=0.00135 $. 所以有$ I/10^{-3}=(0.01575\pm 0.01007)+0.00135L $,故
$$ I_{SC}=(15.75\pm 10.07)+1.35 L $$
结合表3,可以得到不同光照下的填充因子$ FF $,其关系如表5.
\begin{table}[htbp]\small
	\centering
	\caption{不同光照下的填充因子$ FF $}
	\begin{tabular}{cccccc}
%去除原始数据
	\end{tabular}
\end{table}

\subsection{不同负载下硅光电池输出电压$ U $ 与光照 $ L $ 特性测量}
测量不同负载 $ R $的硅光电池输出电压$ U  $与光照 $ L $ 的关系,其电路如图\ref{fig:5}所示. 实验中电压表的量程为$ U=2 $V. 得到的实验数据如表6. 将$ U-L $曲线画出如图\ref{fig:51}.

% Please add the following required packages to your document preamble:
% \usepackage{multirow}
\begin{table}[htbp]\small
	\centering
	\caption{不同负载下硅光电池输出电压$ U $ 与光照 $ L $ 特性测量}
	\begin{tabular}{ccccccccc}
%去除原始数据
	\end{tabular}
\end{table}
% TODO: \usepackage{graphicx} required
\begin{wrapfigure}{r}{0.7\linewidth}
	\centering
	\includegraphics[width=\linewidth]{../实验5图1}
	\caption{不同负载下硅光电池的$ U-L $曲线}
	\label{fig:51}
\end{wrapfigure}

可以看出,随着负载$ R $的增大,$ U $随$L$的增加速度越来越大,即达到一定电压所需要的光照强度越来越小. 且当$ R $变大时,其增加速度是非线性的,在$ R<5000 $Ω时增加速度增大的较快,$ R>5000 $Ω增加速度增大的相较于前就很慢了.
\section{误差分析和讨论}
\subsection{硅光电池输出特性的测量}
在硅光电池输出特性的测量中,由于所使用的电压表量程为$ U=2 $V,当$ R $很小或$ L $时,$ U $的范围都在
20mV以下,容易导致误差过大,而当$ R $很大时,$ U $变化的范围又太小,且已超过了200mV,此时仍存在一定的相对误差。这会使得硅光电池输出特性$ I-U $曲线及$ P-R $曲线在$ R $很小或很大时会稍有不准.

在绘制$ P-R $曲线时,$ R $的刻度采用了$ \log_{10} $的对数处理,因为若采用线性处理,则在$ R<20000 $Ω的曲线会挤在左侧很小的一片区域,不利于观察.
\subsection{硅光电池开路电压$ U_{OC} $ 、短路电流 $ I_{SC}  $ 与光照$  L $ 特性测量}
实验中发现测量短路电流 $ I_{SC}  $时,电压表的读数都在0.02V以下,故使用量程为$ U=200 $mV的电压表测量,使结果更加准确。

可以看出,短路电流 $ I_{SC}  $线性拟合的残差稍大,可能是因为所测量时,接了一个负载$ R=50 $Ω,使得输出电流 $ I $ 随 $ L $ 的增加而缓慢地增加,在达到一定线性范围后则显示为非线性.
\section{思考题}
\kaishu 1.光电池在工作时为什么要处于零偏或反偏?

\songti 当 PN 结反偏时,外加电场与内电场方向一致,耗尽区在外电场作用下变宽,使势垒加强。PN结零偏时,其内部仍存在耗尽层。因此,在它们的结合面耗尽区存在一内电场,当有光照时,电池对光子的本征吸收和非本征吸收都产生光生载流子,形成反向的电流,从而在有负载时检测出形成的电流。

若 PN 结正偏,耗尽区在外电场作用下变窄,使势垒削弱,使载
流子扩散运动继续形成电流,相当于导通了,即使没有光照,也会形成电流,无法检测光照。

\kaishu 2.当增加光照强度,硅光电池的哪些参数发生变化?

\songti 增加光照强度:硅光电池的输出电压和输出电流变化,且基本与光照强度成正比。对应的最大功率$ P_m $随光照强度增加而减小,最加负载电阻$ R_m $岁光照强度增加而增大。填充因子随之变化,但变化是非线性的。开路电压随其增大对数增加,短路电流则线性增加。
\end{document}
