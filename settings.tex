\documentclass[11pt,oneside]{ctexbook} %使书页码在通一遍,避免空白页
\usepackage{ctex}
\usepackage{fontspec}
\usepackage{setspace}
%amsmath
\usepackage{amsmath}
\usepackage{amsthm}
\usepackage{amssymb}
\numberwithin{equation}{section}
\usepackage{graphicx}

\everymath{\displaystyle}
\usepackage{paralist}
\usepackage{enumerate}
\usepackage{wrapfig}
\usepackage{blindtext}
\usepackage{amsfonts}
\usepackage{geometry}
\geometry{left=2.0cm,right=2.0cm,top=2.5cm,bottom=2.5cm}%页边距
%设置数学要素
\usepackage[dvipsnames,svgnames]{xcolor}
\usepackage{thmtools}
\usepackage{hyperref}
\hypersetup{colorlinks=true,linkcolor=Blue}
\setfontfamily{\cons}{Consolas}
\usepackage{listings}
\lstset{
	language = C++,
	backgroundcolor = \color{black!80},
	basicstyle=\cons\color{Khaki},
	breaklines=true,
	numbers=left,
	numberstyle=\sffamily\small\color{black},
	keywordstyle=\color{LightSkyBlue},
	commentstyle=\color{ForestGreen},
	stringstyle=\color{YellowOrange},
	showspaces=false,% 显示添加特定下划线的空格
	showstringspaces=false,%不显示字符串的空格
	showtabs=false, % 在字符串中显示制表符
	%title=\lstname\sffamily,
	xleftmargin=1em,  xrightmargin=1em, % 设定listing左右的空白
	escapeinside={}, % 特殊自定分隔符
	morekeywords = {as}, % 自加新的关键字(必须前后都是空格)
	columns=flexible,
	aboveskip=1ex, belowskip=1ex,
	framextopmargin=1pt, framexbottommargin=1pt,
	abovecaptionskip=-2pt,belowcaptionskip=3pt,
	% 设定中文冲突,断行,列模式,数学环境输入,listing数字的样式
	extendedchars=false, columns=flexible, mathescape=true,
	texcl=true,
	fontadjust
}
